\begin{problem}
    {Q2(a)}
Input: $n$, $\{b_1, \dots, b_n\}$, $L_1, \dots, L_n$
    \begin{algorithmic}[1]
        \STATE Let $G = (V, E)$ be an initially empty graph
        \STATE Suppose the set of buses is denoted $\{b_1, \dots, b_n\}$
        \STATE Suppose the set of drivers is denoted $\{d_1, \dots, d_n\}$
        \STATE Suppose the original matching $(d_i, b_j)$ is denoted as $m(d_i) = b_j$
        \STATE $V = \{s, t\}$
        \FOR {$\forall 0 < i \leq n$}
            \STATE $V = V \cup \{d_i, m(d_i)\}$
            \STATE $L' = L_i - m(d_i)$
            \STATE Add an edge $e = (s, d'_i)$ with cost $0$ and capacity $1$
            \STATE Add an edge $e = (d'_i, m(d_i)$ with cost $0$ and capacity $1$
            \STATE Add an edge $e = (d'_i, d_i)$ with cost $0$ and capacity $1$
            \FOR {$\forall l' \in L'$}
                \STATE Add an edge $e = (d_i, l')$ with cost $1$ and capacity $1$
            \ENDFOR
        \ENDFOR
        \STATE $V = V \cup \{b_{n+1}, d_{n+1}\}$
        \STATE $E = E \cup L_{n+1}$
        \STATE Run any polynomial time algorithm to find a minimum-cost flow for $G$, denoted $f'$ (ie: Successive Shortest Path)
        \IF {$v(f') = n+1$}
        \STATE There exists a matching of buses and drivers
        \ELSE
        \STATE There does not exist a matching of buses and drivers
        \ENDIF
        \IF {$e = (s, m(d_i)) \in f'$}
        \STATE $m(d_i)$ remains matched with $d_i$
        \ELSIF {$e = (d_i, b_j), b_j \neq m(d_i) \in f'$}
        \STATE $d_i$ and $b_j$ are now matched, all other matches involving $d_i, b_j$ are invalidated
        \ENDIF
    \end{algorithmic}
    \noindent
    \textbf{Correctness} \\
    \textbf{Pf. of Validity}
    \begin{proof}
        Note: No driver can be matched with an incompatible bus by construction, since an edge can only exist between a driver and a valid bus. \\
        Claim: No 2 drivers can be matched to the same bus. \\
        Suppose 2 drivers are matched to the same bus. \\
        This means the min-cost flow algorithm $A$ created edges $(d_i, b_k), (d_j, b_k)$ \\
        However, the outgoing flow from $b_k = 1$, and the minimum flow along any edge is $1$ by construction. \\
        Therefore, no two drivers can be assigned to the same bus. \\
        Claim: No driver can be matched to two different buses. \\
        Suppose one driver is matched to 2 different buses. \\
        This means that either an edge goes from $s$ to $m(d_i)$ or two edges leave $d_i$ \\
        The second case is impossible, as the flow entering $d_i$ is bounded by 1 \\
        The first case is impossible, as the flow entering $b'_i$ is bounded by 1 \\\\
        For each driver, he can either stay matched to the same bus or change buses. \\
        This is modeled by the fact that $d'_i$ either sends flow directly to $m(b_i)$ (no change) or $d_i$ (must change by line 8) \\
        Since the algorithm is guaranteed to produce a matching strictly including only matching pairs, it must be valid. \\
    \end{proof}
    \noindent
    \textbf{Pf. of Optimality}
    \begin{proof}
        Suppose the matching generated is non-optimal. \\
        This means there exists some driver $d_i$ which is incorrectly matched to $b_i \neq m(d_i)$ \\
        This means the flow generated by the min-cost flow algorithm has some cost $c \geq 1$ \\
        This means there exists a valid matching where $d_i$ is matched to $m(d_i)$ \\
        To construct an equivalent valid flow to this valid matching, simply take the edge $(d'_i, m(d_i))$ for all unchanged pairs,
        and the edges $(d'_i, d_i), (d_i, b_j)$ for pairs which have changed. \\
        This means there exists a valid flow where $d_i$ is matched to $m(d_i)$ \\
        By line 10, this valid flow has a cost $c' \leq c - 1$ \\
        However, the min-cost flow algorithm is guaranteed to find the lowest cost valid flow. \\
        Contradiction. \\
    \end{proof}
    \noindent
    \textbf{Runtime}
    \begin{proof}
        The construction of $G$ can be done in $O(n^2)$ (polynomial) time. \\
        This is because the algorithm iterates through all $n$ buses/drivers and all $n$ lists, each of which can be up to $n$ long \\
        Solving minimum-cost flow can be done in polynomial time by page 449 in the book. \\
        The final part of the algorithm examines every edge in the flow generated by the aforementioned algorithm. \\
        By construction, there can be up to $3(n+1)$ edges in the flow, because the max path length is $3$ and there are $n+1$ units of flow leaving
        $s$ and entering $t$. \\
    \end{proof}
\end{problem}
