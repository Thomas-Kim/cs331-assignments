\begin{problem}
  {Q5}
  Consider a set of mobile computing clients in a certain town who each need to be connected to one of several possible \textit{base stations}.
  We'll suppose there are $n$ clients, with the position of each client specified by its $(x,y)$ coordinates in the plane. There are also $k$ base stations; the positions
  of each of these is specified by $(x,y)$ coordinates as well. For each client, we wish to connect it to exactly one of the base stations. Our
  choice of connections is constrained in the following ways.\\
  \begin{itemize}
  \item There is a \textit{range parameter} $r$ - a client can only be connected to a base station that is within distance $r$.
  \item There is also a \textit{load parameter} $L$ - no more than $L$ clients can be connected to any single base station.
  \end{itemize}
  Your goal is to design a polynomial-time algorithm for the following problem. Given the positions of a set of clients and a set of base stations, as well as the range and load parameters, decide
  whether every client can be connected simultaneously to a base station, subject to the range and load conditions in the previous paragraph.
  \begin{algorithmic}[1]
    \STATE Create a graph $G(V, E)$
    \STATE Let some edge $e(a, b)_w$ be a directed edge from $a$ to $b$ with weight $w$
    \STATE Let the set of clients be denoted $C = \{c_1, c_2, \dots, c_n\}$
    \STATE Let the set of base stations be denoted $B = \{b_1, b_2, \dots, b_n\}$
    \STATE Define the function $valid(c_i)$ to return some set $B' = \{b'_1, b'_2, \dots, b'_n\} \subset B$ such that $c_i$ is within range $\forall b_i \in B'$
    \STATE $V = \{s, t, c_1, c_2, \dots, c_n, b_1, b_2, \dots, b_n\}$
    \STATE Initially $E = \{\}$
    \FOR {$\forall v_i \in V$}
    \IF {$v_i \in C$}
    \STATE $B' = valid(v_i)$
    \STATE $E = E \cup \{(s, v_i)_1, (v_i, b'_1)_1, \dots, (v_i, b'_n)_1\}$
    \ENDIF
    \IF {$v_i \in B$}
    \STATE $E = E \cup \{(b_i, t)_{L_{b_i}}\}$
    \ENDIF
    \ENDFOR
    \STATE Run Ford-Fulkerson on $G$ to find maximum flow $f$
    \IF {$f = n$}
    \RETURN TRUE
    \ELSE
    \RETURN FALSE
    \ENDIF
  \end{algorithmic}
  \noindent
  \textbf{Correctness}
  \begin{proof}
      The sum of all the edge capacities leaving $s$ is $n$ \\
      If some client $c_i$ cannot be connected to a base station, it is either out of range or overloaded for all base stations. \\
      If it is out of range, there exists no edge between $c_i$ and $b_i$, so there can be no flow between them. \\
      If it is overloaded, no flow can go from $c_i$ to any $b_i$ \\
      Therefore the maximum flow will be less than $n$, as there is no other path from $c_i$ to $t$ by construction. \\
      Therefore, the algorithm will never return false positives. \\
      Suppose Ford-Fulkerson returns $f = n$ when there exists no matching. \\
      This means that some client $c_i$ cannot connect to any base station $b_i$. \\
      However, there exists some $b_i$ that $c_i$ connected to the Ford-Fulkerson algorithm. \\
      By line $14$, $b_i$ is connected to at most $L_{b_i}$ clients. \\
      This means $b_i$ has enough capacity to accept $c_i$. \\
      By line $11$, $b_i$ is in range. \\
      Contradiction. \\
      Therefore, the algorithm returns no false positives and no false negatives. \\
  \end{proof}
  \noindent
  \textbf{Runtime}
  \begin{proof}
      The graph construction takes $n + k$ (polynomial) time, and Ford-Fulkerson runs in polynomial time. \\
      Therefore, the total runtime is polynomial. \\
  \end{proof}
\end{problem}
