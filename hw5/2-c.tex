\begin{problem}
  {Q2(c)}
  Assume $C = 1, 2, 4, \dots, 2^m$ and $N < 2^{m+1}$. For this special case, give an $O(m)$ time algorithm that solves the problem in (a). \\
  \begin{algorithmic}[1]
    \STATE $count = 0$
    \STATE Express $N$ as a binary number $n_1n_2n_3\dots n_k$
    \FOR {$1 \leq i \leq 2^m$}
    \IF {$n_i = 1$}
    \STATE $count := count + 1$
    \ENDIF
    \ENDFOR
  \end{algorithmic}
  \noindent
  \textbf{Correctness}
  \begin{proof}
      Thm: For every element $x \in Q_p$, where $Q_p$ is a field of $p-adic$ integers,
      there exists a unique representation $x = \sum_{i = 0}^{k} a_ip^i, a_k \neq 0$ \\
      By expressing $N$ as a 2-adic integer, it is guaranteed that there exists some
      representation of $N$ in the form shown above. \\
      Note that $k$ in the above sum is bounded by $k < m+1$, since $2^{m+1} > N$ \\
      $\forall 0 \leq i \leq m, 2^i \in C$
      Therefore, there exists a unique representation of $N$ as a sum of the elements in $C$.
  \end{proof}
  \noindent
  \textbf{Runtime}
  \begin{proof}
      Expressing $N$ as a binary number is $O(m)$ time, since $N$ expressed in binary has at most $m$ digits. \\
      Counting the $1$'s in the binary representation of $N$ takes $O(m)$ time, for the above reason.\\
      The complexity is therefore $O(2m) = O(m)$
  \end{proof}
\end{problem}
