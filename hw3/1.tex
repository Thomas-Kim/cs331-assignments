\begin{problem}
  {Q1}
  Prove each of the following heuristics to be ideal or not ideal for scheduling jobs on a machine while minimizing:
  \begin{align*}
      \sum_{i = 1}^{n} w_iC_i
  \end{align*}
  1. Smallest time first. \\
  2. Most important first. \\
  3. Maximum $\frac{w(i)}{t(i)}$. \\\\
  Suppose jobs are expressed as tuples $(w_k, t_k)$ where $w_k$ is the weight of the job and $t_k$ is the time that job takes. \\
  \textbf{Claim:} Smallest time first is \textbf{not} ideal. \\
  \begin{proof}
    Suppose the following set of jobs is scheduled using smallest time first. \\
    $\{j_1 = (1, 2), j_2 = (3, 3)\}$ \\
    The smallest time first heuristic would schedule $j_1$ before $j_2$. \\
    Thus, $j_1$ would finish at time $2$ and $j_2$ would finish at time $5$. \\
    This results in a total cost of $17$. \\
    Suppose $j_2$ is scheduled before $j_1$. \\
    $j_2$ would finish at time $3$ and $j_1$ would finish at time $5$. \\
    This results in a total cost of $14$. \\
    $14 < 17 \implies$ the smallest time first heuristic is not ideal. \\
  \end{proof}
  \textbf{Claim:} Most important first is \textbf{not} ideal. \\
  \begin{proof}
    Suppose the following set of jobs is scheduled using most important first. \\
    $\{j_1 = (2, 3), j_2 = (1, 1)\}$
    The most important first heuristic would schedule $j_2$ before $j_1$. \\
    Thus, $j_1$ finishes at time $3$ and $j_2$ finishes at time $4$. \\
    The total cost is therefore $10$ \\
    Suppose $j_2$ was scheduled before $j_1$. \\
    $j_1$ would finish at time $4$ and $j_2$ would finish at time $1$. \\
    The total cost is therefore $9$. \\
    $9 < 10 \implies$ the most important first heuristic is not ideal. \\
  \end{proof}
  \textbf{Claim:} Maximum $\frac{w(i)}{t(i)}$ first \textbf{is} ideal. \\
  \begin{proof}
      Let the queue of jobs be represented by a set of jobs $\{j_1, j_2, \dots, j_n\}$ where $j_k$ is scheduled before $j_c$ iff $k < c$. \\
      Suppose there exists an inversion between adjacent jobs in the scheduling queue. \\
      This means that for some $k$, $\frac{w_k}{t_k} < \frac{w_{k+1}}{t_{k+1}}$. \\
      This means that $w_kt_{k+1} < w_{k+1}t_k$. \hfill \textbf{(1)} \\
      \textbf{Base case}: The inversion exists between jobs $j_1$ and $j_2$. \\
      $w_1t_1 + w_2(t_1+t_2) > w_2t_2 + w_1(t_1 + t_2)$ \\
      $w_1t_1 + w_2t_1 + w_2t_2 > w_2t_2 + w_1t_1 + w_1t_2$ \\
      $w_2t_1 > w_1t_2$ is true by equation (1) \\
      \textbf{Induction hypothesis}: \\
      $w_{n+1}\left(\sum_{i = 1}^{n+1}t_i\right) + w_{n+2}\left(\sum_{i=1}^{n+2}t_i\right) > w_{n+2}\left(\sum_{i = 1}^{n}t_i + t_{n+2}\right) + w_{n+1}\left(\sum_{i = 1}^{n+2}t_i\right)$ \\
      $w_{n+2}\left(\sum_{i=1}^{n+2}t_i\right) - w_{n+2}\left(\sum_{i = 1}^{n}t_i + t_{n+2}\right) > w_{n+1}\left(\sum_{i = 1}^{n+2}t_i\right) - w_{n+1}\left(\sum_{i = 1}^{n+1}t_i\right) $ \\
      $w_{n+2}\left(\sum_{i = 1}^{n+2}t_i - \sum_{i = 1}^{n}t_i - t_{n+2}\right) > w_{n+1}\left(\sum_{i = 1}^{n+2}t_i - \sum_{i = 1}^{n+1}t_i\right)$ \\
      $w_{n+2}\left(t_{n+1} + t_{n+2} - t_{n+2}\right) > w_{n+1}t_{n+2}$ \\
      $w_{n+2}t_{n+1} > w_{n+1}t_{n+2}$ \\
      This is true by equation (1). \\
  \end{proof}
\end{problem}
