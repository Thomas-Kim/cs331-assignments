\begin{problem}
  {Q2(b)}
  Show that it possible to guess a number 1-N given 3 "strikes" (questions answered no) and $O(n^{\frac{1}{3}})$ questions. \\
  \begin{algorithmic}[1]
    \STATE Phase 1: $g_1 = 0$, Guessing number $1 < n < N$
    \STATE Define a mapping $f(x) \rightarrow x^{\frac{1}{3}}$
    \FOR {$g_1 = 0 \dots N^{\frac{1}{3}}$}
    \STATE Ask "Is $f(n) > g_1$?"
    \IF {Answer = "No"}
    \STATE Save the current value of $g_1$ and go to Phase 2
    \ENDIF
    \ENDFOR
    \STATE Phase 2: $g_2 = 0$
    \STATE Define a mapping $g(x) \rightarrow (x - (g_1 - 1)^3)$
    \FOR {$g_2 = 0 \dots \sqrt{3g_1^2 - 3g_1 + 1}$}
    \STATE Ask "Is $(g(f(n)) > g_2$?"
    \IF {Answer = "No"}
    \STATE Save the current value of $g_2$ and go to Phase 3
    \ENDIF
    \ENDFOR
    \STATE Phase 3: $g_3 = 0$
    \STATE Define a mapping $h(x) \rightarrow (x - (g_2 - 1)^2)$
    \FOR {$g_3 = 0 \dots 2g_2 - 1$}
    \STATE Ask "Is $h(g(x)) > g_3$?"
    \IF {Answer = "No"}
    \STATE Guess $h^{-1}(g^{-1}(x))$
    \ENDIF
    \ENDFOR
  \end{algorithmic}
  \textbf{Lemma 1a:} At the beginning of phase 2, up to $N^{\frac{1}{3}}$ questions will have been asked. \\
  \begin{proof}
    By line 3, phase 1 is guaranteed to terminate after the loop finishes, which finishes in $N^{\frac{1}{3}}$ iterations. \\
    By line 4, during each iteration of the loop in phase 1, precisely 1 question is asked. \\
    $\implies $ no more than $N^{\frac{1}{3}}$ questions can be asked during phase 1. \\
    By line 6 and line 8, phase 1 can only directly precede phase 2. \\
    $\implies $ Phase 2 immediately follows phase 1, so no questions can be asked between phase 1 and phase 2 \\
    $\implies $ At the beginning of phase 2, up to $N^{\frac{1}{3}}$ questions can have been asked. \\
  \end{proof}
  \textbf{Lemma 1b:} At the beginning of phase 2, precisely 1 strike will be used. \\
  \begin{proof}
    By line 6, if 1 strike is used during phase 1, the algorithm will immediately go to phase 2. \\
    Suppose no strikes are used during phase 1. \\
    $\implies n^{\frac{1}{3}} > N^{\frac{1}{3}}$ by lines 2 and 3, since $g_1 = N^{\frac{1}{3}}$ during the last iteration of the loop. \\
    $\implies n > N$, which means $n$ is not in the range 1-N. \\
    Therefore, no more than 1 strikes can be used during phase 1, and at least 1 strike must be used during phase 1 \\
    Phase 1 always leads immediately to phase 2, from the proof of lemma 1. \\
    $\therefore $ precisely 1 strike will be used at the beginning of phase 2. \\
  \end{proof}
  \textbf{Lemma 2:} At the beginning of phase 2, the possible values of $n$ will be bounded by $(g_1 - 1)^3 < n \leq (g_1)^3$ \\
  \begin{proof}
    By lemma 1b, precisely 1 strike must have been used during phase 1. \\
    $\implies $ line 5 was executed precisely 1 time. \\
    Suppose $g_1 = \alpha$ when line 5 was executed. \\
    During the previous iteration of the loop, line 5 must not have executed, because otherwise the loop would terminate. \\
    During the previous iteration of the loop, $g_1 = \alpha - 1$. \\
    Since line 6 did not execute during the previous iteration, $\alpha - 1 < n^{\frac{1}{3}}$ by lines 3 and 4. \\
    Since line 6 executed during this iteration, $\alpha \geq n^{\frac{1}{3}}$ by line 4 and 6. \\
    $\implies (\alpha - 1)^3 < n \land (\alpha)^3 \geq n \implies (\alpha - 1)^3 < n \leq (\alpha)^3$ \\
    By line 3 of this proof, $g_1 = \alpha \implies (g_1 - 1)^3 < n \leq (g_1)^3$ \\
  \end{proof}
  \textbf{Lemma 3a:} During phase 2, up to $\sqrt{3N^{\frac{2}{3}} - 3N^{\frac{1}{3}} + 1}$ questions will be asked
  \begin{proof}
    By \textbf{Lemma 2} and the definition of $g(x)$, $ 0 < g(n) < \sqrt{3g_1^{\frac{2}{3}} - 3g_1^{\frac{1}{3}} + 1}$ \\
    By line 3 of the algorithm, $g_1$ is bounded by $N^{\frac{1}{3}}$ \\
    By line 11 of the algorithm, the phase 2 loop iterates $\sqrt{3g_1^{\frac{2}{3}} - 3g_1^{\frac{1}{3}} + 1}$ times \\
    $\therefore \sqrt{3g_1^{\frac{2}{3}} - 3g_1^{\frac{1}{3}} + 1}$ is bounded by $\sqrt{3N^{\frac{2}{3}} - 3N^{\frac{1}{3}} + 1}$ \\
  \end{proof}
  \textbf{Lemma 3b:} At the beginning of phase 3, precisely 2 strikes will be used. \\
  \begin{proof}
    By line 14, if 1 strike is used during phase 2, the algorithm will immediately go to phase 3. \\
    Suppose no strikes are used during phase 2. \\
    This means during the last iteration, $g_2 = (g_1)^3 - (g_1 - 1)^3 \land f(n) > g_2$ \\
    By \textbf{lemma 2}, $(g_1 - 1)^3 < n \leq (g_1)^3 \implies 0 < n - (g_1 - 1)^3 < (g_1)^3 - (g_1 - 1)^3$ \\
    By line 11, the algorithm iterates $(g_1)^3 - (g_1 - 1)^3$ times, iterating through $0 < g_2 \leq (g_1)^3 - (g_1 - 1)^3$ \\
    By line 3 of this proof, $0 < f(n) < (g_1)^3 - (g_1 - 1)^3$ \\
    $\therefore $ if $f(n) > g_2$ during the last iteration, $f(n) > (g_1)^3 - (g_1 - 1)^3 \implies n > (g_1)^3$ \\
    $\implies $ at least 1 strike must be used during phase 2. \\
    By line 8 of this proof, and \textbf{lemma 1b}, precisely 2 strikes must be used by the end of phase 2 \\
    Since phase 2 always immediately leads to phase 3, precisely 2 strikes have been used at the beginning of phase 3. \\
  \end{proof}
  \textbf{Lemma 4:} At the beginning of phase 3, the possible values of $g(h(n))$ will be bounded by $0 < g(h(n)) \leq 2(\sqrt{3(N^{\frac{2}{3}} - N^{\frac{1}{3}}) + 1}) - 1$ \\
  \begin{proof}
    At the beginning of phase 3, the possible vales of $h(n)$ will be bounded by $(g_2 - 1)^2 < h(n) \leq (g_2)^2$ \\
    Subtract $(g_2 - 1)^2$ from all sides of the inequality. \\
    Note that $g_1 \leq N^{\frac{1}{3}} \land g_2 \leq \sqrt{(g_1)^3 - (g_1 - 1)^3}$ \\
    $0 < g(h(n)) \leq 2(\sqrt{3(N^{\frac{2}{3}} - N^{\frac{1}{3}}) + 1}) - 1$ \\
  \end{proof}
  \textbf{Lemma 5:} $g^{-1}(x)$ and $h^{-1}(x)$ are well defined and bijective.
  \begin{proof}
    $g^{-1}(x) = (x + (g_1 - 1)^3)$ \\
    $h^{-1}(x) = (x + (g_2 - 1)^2)$ \\
    Both are linear, and are therefore bijective. \\
  \end{proof}
  \textbf{Proof of correctness:}
  \begin{proof}
  By \textbf{Lemma 4}, the possible values of $n$ are bounded by $(\beta - 1)^3 \leq n \leq (\beta)^3$. \\
  By line 19, 20 of the algorithm, Phase 3 is a linear search over this range \\
  By \textbf{Lemma 5}, $g_3 - 1 < n \land n \leq g_3 \implies n = g_3$ \\
  By line 22 of the algorithm, the value guessed is $h^{-1}(g^{-1}(g_3)) = n$. \\
  \end{proof}
  \textbf{Proof of termination within $O(N^{\frac{1}{3}})$ time:}
  \begin{proof}
    By lemma 1a Phase 1 uses $N^{\frac{1}{3}}$ questions. \\
    By lemma 2a Phase 2 uses $\sqrt{3N^{\frac{2}{3}} - 3N^{\frac{1}{3}} + 1}$ questions. \\
    By lemma 4 and the fact that phase 3 is linear search, Phase 3 uses $2(\sqrt{3(N^{\frac{2}{3}} - N^{\frac{1}{3}}) + 1}) - 1$ questions. \\
    No term grows faster than $N^{\frac{1}{3}}$ \\
  \end{proof}
  \textbf{Proof of termination using precisely 3 strikes:}
  \begin{proof}
    By \textbf{Lemma 3}, At the beginning of phase 3, precisely $2$ strikes have been used up. \\
    By line 21 of the algorithm, after receiving the $3^{rd}$ strike, the algorithm will halt. \\
  \end{proof}
\end{problem}
