\begin{problem}
  {Q3}
    Modify the Gale-Shapley Algorithm to fulfill the following definition of stability, and prove correctness \\
    Stability implies none of the following types of instability: \\
    \begin{itemize}
      \item Type 1 instability: There is a candidate $X$ and a team $Y$ , such that
        $X$ is not assigned to $Y$, but $X$ prefers $Y$ more than his assigned team,
        and $Y$ prefers $X$ more than at least one of the candidates assigned
        to it.
      \item Type 2 instability: There exists a candidate $X$ who is not assigned to
        any team, but there is a team $Y$ with an empty spot, or a candidate
        $X_0$ assigned to it such that $Y$ prefers $X$ more than $X_0$.
      \item Type 3 instability: There exists a team with one or more empty
        spots, but there is an unassigned candidate.
    \end{itemize}
    Suppose the set of teams $G = \{T_1, T_2, \dots, T_n\}$ where each team is a set of empty spots $T_i = \{T_{i, 1}, T_{i, 2}, \dots, T_{i, m}\}$ \\
    %Let the set of men $M = \{T_{1, 1}, \dots, T_{1, m}, \dots T_{n, m}\}$ such that $\forall j, k, T_{x, j} \in M, T_{x, j} \not<_{y} T_{x, k}$ \\
    Let the set of men $M$ be the set of empty slots. \\
    Let the set of women $W$ be the set of candidates. \\
    The preference list for each empty slot is the same as the preference list for its respective team. \\
    For each candidate, expand each team $T_{i}$ on their preference list into a list of slots $\{T_{i, j}\}$, ordered such that lower $j$ are more preferred \\
    Now simply run the modified G-S algorithm on these two sets $M$ and $W$ as shown below.
    \begin{algorithmic}[1]
      \STATE Pick one unmatched man $x$
      \IF {$x$ has not yet proposed to a woman in his list}
        \STATE $x$ proposes to the top woman $y$ in his list who he has not yet proposed to
      \ELSE
        \STATE mark $x$ done
        \STATE goto 1
      \ENDIF
      \IF {$y$ is free}
        \STATE $y$ temporarily accepts $x$
      \ELSIF {$y$ is matched with $x' \ne x$}
        \IF {$x <_{y} x'$}
        \STATE $y$ accepts $x$ temporarily, $x'$ becomes free
        \ELSE
        \STATE $y$ rejects $x$
        \ENDIF
      \ENDIF
      \STATE Repeat until every man is marked done or is matched
    \end{algorithmic}
    \textbf{Lemma 1:} The G-S algorithm will either match every woman to a man or every man to a woman given 2 sets of unequal length
    \begin{proof}
      Suppose $|W| > |M|$. \\
      The modified G-S algorithm will not terminate until every man is matched or has proposed to every woman on his list. \\
      Suppose some man $m$ is unmatched when the algorithm completes. \\
      If $m$ is unmatched, he must been rejected by every woman $w \in W$ \\
      Every woman $w \in W$ must already be matched to a man $m': m' >_{w} m$ \\
      However, $|W| > |M|$, so there must be some woman who is not matched to a man. \\
      $\therefore$ if $|W| > |M|$, every man must be matched by the end of the modified G-S algorithm. \\\\
      %%%%%%%%%%%%%%%%%%%%%%%%%%%
      Suppose $|M| > |W|$. \\
      The modified G-S algorithm will not terminate until every man is matched or has proposed to every woman on his list. \\
      Suppose some woman $w$ is unmatched when the algorithm completes. \\
      If $w$ is unmatched, she must not have been proposed to by any man $m \in M$ \\
      If a woman has not been proposed to, every man must already be matched. \\
      However, $|M| > |W|$, so there must exist some $m \in M$ which is unmatched. \\
      $\therefore$ if $|M| > |W|$, every woman must be matched by the end of the modified G-S algorithm. \\
    \end{proof}
    Fact 1: The set $S_{G-S}$ resulting from running the G-S algorithm will never contain an unstable pair, where instability is defined as $(m,w) \in S_{G-S} : \exists m', w', m <_{w} m' \land w <_{m} w'$ \\
    \noindent
    \textbf{Proof of Type 1 instability:}
    \begin{proof}
    \end{proof}
    \noindent
    \textbf{Proof of Type 2 instability:}
    \begin{proof}
    \end{proof}
    \noindent
    \textbf{Proof of Type 3 instability:}
    \begin{proof}
    By \textbf{lemma 1}, the modified G-S algorithm must either match all women or all men. \\
    \textbf{Case 1:} Suppose the modified G-S algorithm matches all men but not all women. \\
    By definition, this means all of the slots are matched and there are some unassigned candidates. \\
    By definition, this means no team exists with an empty slot. \\
    Therefore, there cannot exist a team with an empty slot if there are some unassigned candidates. \\
    \textbf{Case 2:} Suppose the modified G-S algorithm matches all women but not all men. \\
    This means there exist no unmatched candidates and there are some empty slots. \\
    By definition, this means some team exists with an empty slot. \\
    Therefore, there cannot exist unmatched candidates if some team exists with an empty slot. \\
    \textbf{Case 3:} Suppose all men and women are matched. \\
    By definition, this means no team has an empty slot and there are not unassigned candidates. \\
    \end{proof}
\end{problem}
