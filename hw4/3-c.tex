\begin{problem}
  {Q3(c)}
  Let the set of independent vertices $S_i \subset \{v_1,v_2,v_3,\dots,v_i\}, i \leq n$ have some weight $W_i$ \\
  Let each vertex $v_i$ have some weight $w_i$ \\
  Suppose $S_i$ includes vertex $v_i$. $S_i$ can therefore cannot include $v_{i-1}$, so $W_i = W_{i-2} + w_i$ \\
  Suppose $S_i$ does not include $v_i$. $W_i = W_{i-1}$ \\
  \textbf{Base case(s)}: $W_0 = 0, W_1 = w_1$ \\
  \textbf{Algorithm}:
  $W_i = max(W_{i-1}, W_{i-2} + w_i)$ \\
  return $W_n$ \\
  \textbf{Memoization}: $W_j$ only has to be computed once for each unique $j$. \\
  \textbf{Proof of correctness}:
  \begin{proof}
    Suppose $W_n$ is not the maximum weight. \\
    This means either $W_{i-1}$ or $W_{i-2}$ wasn't maximized. \\
    This continues, with some $W_{i - k}$ not being maximized. \\
    However, $W_0$ and $W_1$ are always maximized by definition. \\
    Thus, $W_{i - k}, k = i$ must be maximized. \\
    This contradicts the original assumption, since each step is guaranteed to maximize the weight if the previous step is maximized. \\
  \end{proof}
  \textbf{Proof of runtime}:
  \begin{proof}
    $W_0, W_1, \dots, W_n$ will all be calculated exactly once due to memoization. \\
    The time required to calculate $W_i$ is $O(1)$, since it is a single comparison and a single addition. \\
    Therefore, the total complexity is $O(n)$. \\
  \end{proof}
\end{problem}
