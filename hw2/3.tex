\begin{problem}
  {Q3}
  Given $n$ images and $m$ unambiguous matches, design an algorithm that runs in $O(m+n)$ time and
  uniquely labels $n$ images as either A or B, such that two images reported to be the same by
  ImgComp get the same label, and two images reported to be different by ImgComp get different labels. \\
  Prove the algorithm's runtime and correctness. \\
  \begin{algorithmic}[1]
    \STATE \textbf{Phase 1}
    \STATE Let $I$ be the set of images such that $|I| = n$ \\
    \STATE Let $U$ be the set of unambiguous matches such that $|U| = m$ \\
    \FOR {$\forall i_x \in I$}
    \STATE Create a node $v_x$ corresponding to $i$ in graph $G$
    \ENDFOR
    \FOR {$\forall (i_x, i_y) \in U$}
    \IF {$i_x$ is DIFFERENT from $i_y$}
    \STATE Create an edge between the nodes $v_x$ and $v_y$ in graph $G$
    \ENDIF
    \ENDFOR
    \STATE $T = \emptyset$
    \WHILE {$\exists s \in G$ such that $s$ has not been explored by BFS}
    \STATE Execute BFS starting at $s$ to generate subtree $t_j$
    \STATE $T = T \cup \{t_j\}$
    \ENDWHILE
    \STATE \textbf{Phase 2}
    \STATE Let the current color $c = 1$
    \FOR {$\forall t_j \in T$}
    \IF {$t_j$ has at least 1 edge}
      \STATE Color all odd-number layered nodes in $t_j$ using color $c$
      \STATE Color all even-number layered nodes in $t_j$ using color $c + 1$
      \STATE $c = c + 2$
    \ENDIF
    \ENDFOR
    \STATE \textbf{Phase 3}
    \STATE Let $S = \emptyset$
    \STATE Choose some arbitrary node which has an assigned color $c_x$
    \STATE $S = S \cup \{c_x\}$
    \FOR {$\forall (i_x, i_y) \in U \mid| c_x \in S$}
    \IF {$i_x$ is the SAME as $i_y$}
    \STATE Let $c_y$ be the color of $i_y$
    \IF {$c_y$ is not defined}
      \STATE $c_y = c + 1$
      \STATE $c = c + 1$
    \ENDIF
    \STATE $S = S \cup \{c_y\}$
    \STATE Remove $(i_x, i_y)$ from $U$
    \ENDIF
    \ENDFOR
    \STATE \textbf{Phase 4}
    \FOR {$i \in G$}
    \IF {$c_i \in S$}
    \STATE Label $i$ with A
    \ELSE
    \STATE Label $i$ with B
    \ENDIF
    \ENDFOR
  \end{algorithmic}
  \textbf{Proof of runtime: }
  \begin{proof}
    The runtime of phase 1 is $O(2m + 2n)$, as the graph construction takes $O(m+n)$ time and BFS takes $O(m+n)$ time. \\
    The runtime of phase 2 is $O(m+n)$, as the graph coloring can be done using a BFS-like algorithm. \\
    The runtime of phase 3 is $O(m)$, as each unambiguous match is explored once. \\
    The runtime of phase 4 is $O(n)$, as each node is explored once to label it. \\
    The total runtime is therefore $O(2m + m + m + 2n + n + n) = O(4m + 4n) = O(m + n)$ \\
  \end{proof}
  \textbf{Lemma 1:} There must be no odd-length cycles in $G$.
  \begin{proof}
    Suppose there is an odd-length cycle in $G$.\\
    This means $G$ is not 2-colorable. \\
    This means there exists no way to label each image as either $A$ or $B$ fitting the conditions. \\
  \end{proof}
  \textbf{Lemma 2:} By the end of phase 3, precisely all of the colors which correspond to matching nodes (to the arbitrarily chosen color) must be part of $S$.
  \begin{proof}
    Suppose some color $c \in S, c$ is different from $c_i \in S$. \\
    $\implies \exists (i_x, i_y) \in U \mid| c_x \ne c_y \land i_x$ and $i_y$ are the same by lines 30, 31 \\
    $\implies c_x, c_y$ are the same \\
    $\rightarrow$ contradiction. \\
  \end{proof}
  \textbf{Proof of correctness: }
  \begin{proof}
    By the end of phase 2, every node will be colored such that no node shares the same color with a node it is known to be different from. \\
    Suppose some node is colored such that it shares the same color as a node it is different from. By lines 21 and 22, these nodes must be on the same layer of $T$. \\
    By lemma 1, there must be no odd-length cycles in $T$. \\
    If there are no odd-length cycles in $T$, there cannot exist a non-tree edge connecting two nodes in the same layer. \\
    If there is no non-tree edge connecting two nodes in the same layer, by the definition of $T$ they cannot be different. \\
    By lemma 2, at the end of phase 3, precisely all of the colors which correspond to matching nodes must be part of $S$.
    During phase 4, all colors in $S$ are labeled A, so precisely all of those colors which match the arbitrarily chosen color from line 28 are labeled A. \\
    The other nodes are all either known to be different from A, or are ambiguous to all other nodes, thus can safely be labeled B. \\
    Suppose a node has not been assigned a color by the end of phase 3.
    This node must not be different from any node by phase 2 lines 21-22. \\
    This node must not be same as any node by phase 3 lines 33-35. \\
    Thus, this node is ambiguous to every other node and can be safely colored using either A or B. \\
  \end{proof}
\end{problem}
