\begin{problem}
    {Q4: 8.22}
    Suppose that someone gives you a black-box algorithm $\mathcal{A}$ that takes an undirected graph $G = (V,E)$ and a number $k$, and behaves as
    follows:
    \begin{itemize}
      \item If $G$ is not connected, it simply returns "$G$ is not connected"
      \item If $G$ is connected and has an independent set of size at least $k$, it returns "yes"
      \item If $G$ is connected and does not have an independent set of size at least $k$, it returns "no"
    \end{itemize}
    Suppose that the algorithm $\mathcal{A}$ runs in time polynomial in this size of $G$ and $k$.

    Show how, using calls to $\mathcal{A}$, you could then solve the Independent Set Problem in polynomial time: Given an arbitrary undirected
    graph $G$, and a number $k$, does $G$ contain an independent set of size at least $k$?
    \begin{algorithmic}[1]
        \STATE Use BFS to find a set of connected subgraphs of $G$ $\mathcal{G} = \{G_1, G_2, \dots, G_m\}$
        \STATE $SUM := 0$
        \FOR {$\forall G_i \in \mathcal{G}$}
            \STATE $MAX := 0$;
            \FOR {$\forall 1 \leq j \leq |G_i|$}
                \IF {$\mathcal{A}(G_i, j) = $ "yes"}
                    \STATE $MAX := j$
                \ENDIF
            \ENDFOR
            \STATE $SUM := SUM + MAX$
        \ENDFOR
        \IF {$SUM \geq k$}
        \RETURN $"YES"$
        \ELSE
        \RETURN $"NO"$
        \ENDIF
    \end{algorithmic}
    \textbf{Runtime}
    \begin{proof}
        BFS can be done in polynomial time $O(|E|)$. \\
        $\mathcal{A}$ is called on each connected component up to $|V|$ times. \\
        This is $O(|V|^2)$ time, which is polynomial. \\
        Therefore, the total runtime is polynomial. \\
    \end{proof}
    \textbf{Correctness}
    \begin{proof}
        The algorithm identifies the maximum size independent set for each connected component. \\
        By definition, different connected subgraphs do not share any edges. \\
        Therefore, the union of two independent sets for two different connected components is an independent set
        for the graph composed of these two components. \\
        By line 5, only the largest independent set for each connected component is used in the final calculation. \\
        By lines 7 and 10, the algorithm sums each of the independent sets for each connected component. \\
        By the previous claim, the union of these sets is a valid independent set for each connected component. \\
        By the previous claim, the largest independent set for the entire graph is the union of the aforementioned sets. \\
        By line 12, this cardinality for the largest independent set is compared to $k$. \\
        Therefore, the algorithm will always accurately decide the Independent Set Problem, since it always finds the largest independent set. \\
    \end{proof}
\end{problem}
