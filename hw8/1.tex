\begin{problem}
    {Q1: 13.2}
    Consider a county in which 100,000 people vote in an election.
    There are only two candidates on the ballot: a Democratic candidate (denoted $D$) and
    and Republican candidate $R$. As it happens, this county is heavily Democratic, so 80,000
    people go to the polls with the intention of voting for $D$, and 20,000 go to the polls with the intention of voting for $R$.\\

    However, the layout of the ballot is a little confusing, so each voter, independently and with probability $\frac{1}{100}$, votes for the
    wrong candidate - that is, the one that he or she \textit{didn't} intend to vote for.
    (Remember that in this election, there are only two candidates on the ballot.)\\

    Let $X$ denote the random variable equal to the number of votes received by the Democratic candidate $D$ when the voting is conducted with this
    process of error. Determine the expected value of $X$, and give an explanation of your derivation of this value.
\end{problem}
