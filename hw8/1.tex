\begin{problem}
    {Q1: 13.2}
    Consider a county in which 100,000 people vote in an election.
    There are only two candidates on the ballot: a Democratic candidate (denoted $D$) and
    and Republican candidate $R$. As it happens, this county is heavily Democratic, so 80,000
    people go to the polls with the intention of voting for $D$, and 20,000 go to the polls with the intention of voting for $R$.\\

    However, the layout of the ballot is a little confusing, so each voter, independently and with probability $\frac{1}{100}$, votes for the
    wrong candidate - that is, the one that he or she \textit{didn't} intend to vote for.
    (Remember that in this election, there are only two candidates on the ballot.)\\

    Let $X$ denote the random variable equal to the number of votes received by the Democratic candidate $D$ when the voting is conducted with this
    process of error. Determine the expected value of $X$, and give an explanation of your derivation of this value.

    \begin{proof}
        Let $A_i$ be a binary random variable where $A_i = 1$ if voter $i$ voted for $D$ and $A_i = 0$ if voter $i$ voted for $R$. \\
        By this, $X = \sum_{1}^{100000} A_i$ \\
        For $1 \leq i \leq 80000$, $E[A_i] = 0.99 \times 1 + 0.01 \times 0 = 0.99$ \\
        For $80000 < i \leq 100000$, $E[A_i] = 0.99 \times 0 + 0.01 \times 1 = 0.01$ \\
        Since each event is independent, the cumulative expectation is linear with respect to each individual event. \\
        Therefore, $E[X] = \sum_{1}^{80000} 0.99 + \sum_{80001}^{100000} 0.01 = 79400$ \\
    \end{proof}
\end{problem}
