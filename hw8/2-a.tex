\begin{problem}
  {Q2(a): 13.11(a)}
  \textit{Load balancing algorithms} for parallel or distributed systems seek to spread out collections of computing
  jobs over multiple machines. In this way, no one machine becomes a "hot spot." If some kind of central coordination is possible, then the load can
  potentially be spread out almost perfectly. But what if the jobs are coming from diverse sources that can't coordinate? As we saw in
  Section 13.10, one option is to assign them to machines at random and hope that this randomization will work to prevent imblanaces. Clearly , this
  won't generally work as well as a perfectly centralized solution, but it can be quite effective. Here we try analyzing some variations and extensions
  on the simple load balancing heuristic we considered in Section 13.10.
  
  Suppose you have $k$ machines, and $k$ jobs show up for processing. Each job is assigned to one of the $k$ machines independently at random (with
  each machine equally likely).

  Let $N(k)$ be the expected number of machines that do not receive any jobs, so that $\frac{N(k)}{k}$ is the expected fraction of machines with
  nothing to do. What is the value of the limit $lim_{k \rightarrow \infty}\frac{N(k)}{k}$? Give a proof of your answer.
\end{problem}
