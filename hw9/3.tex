\begin{problem}
    {Q3: 8.17}
    \textit{Zero-Weight-Cycle Problem}: \\
    Given a directed graph $G = (V,E)$ with weights $w_e \in \mathbb{Z}$, does there exist a simple cycle $C$ such that $\sum_{e \in C}w_e = 0$? \\
    Prove NP-Completeness \\
    \textbf{Proof of NP}
    \begin{proof}
        Let the certificate be some cycle $c = \{e_1, e_2, \dots, e_n\}$ in G \\
        To verify whether the solution is valid, check whether $\sum_{i = 1}^{n}w_i = 0$ \\
    \end{proof}
    \textbf{Reduction from Subset-Sum to Zero-Weight-Cycle}
    \begin{algorithmic}[1]
        \STATE Let $T, A = \{a_1, a_2, \dots, a_n\}$ be an instance of the subset-sum problem with target sum $T$ and set $A$
        \STATE Let $G = (V, E)$ be an initially empty graph
        \STATE Let $e = (x, y)_w$ be a directed edge from $x$ to $y$ with weight $w$
        \STATE $V := \{s, v_1, v_2, \dots, v_n\}$
        \FOR {$\forall a_i \in A$}
            \FOR {$\forall 1 \leq j < i$}
                \STATE $E := E \cup \{(v_j, v_i)_{a_i}\}$
            \ENDFOR
            \STATE $E := E \cup \{(v_i, s)_{-T}\}$
        \ENDFOR
        \RETURN $Zero-Weight-Cycle(G)$
    \end{algorithmic}
    \noindent
    \textbf{Proof of correctness}:
    \begin{proof}
        Claim: Any cycle $\{e_1, e_2, \dots, e_n\}$ contains exactly one or zero edges corresponding to each weight $a_i \in A$
        Suppose 2 edges in the cycle have the same weight $a_i$. \\
        This means 2 edges entered $v_i$, as only edges incoming to $v_i$ have weight $a_i$ \\
        This means $v_i$ was used at least twice, thus it is not a simple cycle. \\
        As a corollary, this means every cycle corresponds precisely to one valid subset of $A$ \\\\
        Claim: Any simple cycle must pass through $s$. \\
        Remove all edges with one endpoint at $s$ from the graph \\
        By construction, there exists no edge between nodes $v_i, v_j$ where $i > j$ \\
        Therefore, there exists a topological ordering of this new graph. \\
        Therefore, there exist no cycles in this new graph without any edges to $s$. \\
        Therefore, any cycle must pass through $s$. \\\\
        Claim: No edge in the graph has a weight $w \neq a_i \land w \neq -T$ \\
        By construction, every graph is assigned a weight either from $a_i \in A$ or $-T$ \\\\
        Claim: If a simple cycle $C$ has weight 0, there exists a subset with sum $T$ \\
        Remove the edge $(v_k, s)$ from the cycle to get $C'$. 
        By construction, $\sum_{e_i \in C'}w_i = T$ since the weight of $(v_k, s) = -T$ \\
        Construct a subset $A'$ which contains all $a_i$ such that $e_i \in C'$ \\
        The sum of this subset is precisely $T$ \\
        Claim: If no simple cycle $C$ has weight 0, there exists no subset with sum $T$ \\
        Suppose there existed a subset $\{x_1, x_2, \dots, x_n\}$ with sum $T$. \\
        Order this set such that $x_i < x_j \equiv i < j$ \\
        Create a cycle using edges $(s, x_1), (x_1, x_2), \dots, (x_n, s)$ \\
        The first $n$ edges have a total weight of $T$ by construction. \\
        The last edge has a weight of $-T$ \\
        Therefore the cycle has zero weight. \\
        Contradiction. \\
    \end{proof}
    \noindent
    \textbf{Proof of runtime}:
    \begin{proof}
        Let the size of the set be $N$ \\
        For each set element, precisely 1 vertex and at most $\theta(N)$ edges are made \\
        Therefore, the runtime of the reduction is $\theta(N^2)$ \\
        This is polynomial in $N$. \\
    \end{proof}
\end{problem}
